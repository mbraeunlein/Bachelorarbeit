\begin{abstract}
Das Zahlenrätsel Sudoku ist weltweit bei Rätsel-Liebhabern bekannt und beliebt. Seit seiner Veröffentlichung 1986 begeistern sich immer mehr Menschen für mehr oder weniger schwierige Exemplare. Sudokus finden sich im Internet, in der Rätselecke der Tageszeitungen und sogar als ganze Bücher, um nur einige Erscheinungsorte zu nennen. \\
Die Regeln sind einfach zu lernen und doch kann man sich sehr lange mit Sudokus beschäftigen, da die schwersten Sudokus meist nur von Profis gelöst werden können.\\
Der Spielspaß ist sehr stark davon abhängig, dass die Schwierigkeit zum persönlichen Können passt. Ist das Sudoku zu leicht, stellt es keine Herausforderung dar. Ist es zu schwer, kommt schnell ein Gefühl der Überforderung auf. \\
Die ausgewiesenen Schwierigkeitsstufen von Sudokus aus verschiedenen Quellen haben zwar oft die gleichen Namen wie zum Beispiel 
\textquotedblleft Mittel\textquotedblright
, unterscheiden sich aber dennoch häufig nach Meinung des Spielers.\\
Das Ziel dieser Bachelorarbeit ist, Merkmale aus Sudokus zu extrahieren, anhand derer die Sudokus von einem Klassifizierer möglichst zuverlässig in Schwierigkeitsstufen eingeteilt werden können.\\
\end{abstract}