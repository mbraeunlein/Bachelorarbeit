\begin{abstract}
Diese Arbeit beschäftigt sich mit der Frage, inwiefern es möglich ist, Sudokus von einem Klassifikationsalgorithmus in Schwierigkeitsgrade einteilen zu lassen. Nach einer Einführung in die Begrifflichkeiten werden Lösungsmethoden für Sudokus vorgestellt, die im Rahmen dieser Arbeit verwendet wurden. Anschließend wird die entwickelte Methode beschrieben, die aus Sudokus geschickt sogenannte Featurevectoren extrahiert und mit Trainigsbeispielen einen Klassifizierer trainiert. Danach wird eine Vorgehensweise beschrieben, mit der eine Abbildung von zwei Sudoku Bewertungsskalen erstellt werden kann. Im Anschluss wird die entwickelte Software zur praktischen Umsetzung vorgestellt. Die Ergebnisse, die erzielt wurden, werden ausführlich diskutiert und bewertet. Im letzten Kapitel erfolgt eine Zusammenfassung und ein Ausblick auf mögliche, weitere Forschung.
\end{abstract}