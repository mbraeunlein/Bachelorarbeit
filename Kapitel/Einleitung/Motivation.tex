\section{Motivation}
Das Zahlenrätsel Sudoku ist weltweit bei Rätsel-Liebhabern bekannt und beliebt. Nach seiner Veröffentlichung 1979 dauerte es bis 1986, bis sich Sudoku wie ein Lauffeuer über den Globus verbreitete. Seitdem begeistern sich immer mehr Menschen für mehr oder weniger schwierige Exemplare. Sudokus finden sich im Internet, in der Rätselecke der Tageszeitungen und sogar als ganze Bücher, um nur einige Erscheinungsorte zu nennen. \\
Laut \cite{SuDra:2014:Misc} kommt das Wort Sudoku ursprünglich aus dem japanischen, 'Su' bedeutet 'Zahl' und 'Doku' bedeutet 'einzeln'. Weiter liest man dort, dass der Grundstein zu Sudokus von dem Mathematiker Leonhard Euler gelegt worden sei. Das Puzzle sei über Frankreich und Amerika nach Japan gewandert, wobei es sich kontinuierlich von lateinischen Quadraten\footnote{\url{http://www.statistik.tuwien.ac.at/public/dutt/vorles/statistII_10/compstat/node33.html}} zu dem heute bekannten Sudoku entwickelt habe. Die erste Veröffentlichung von Sudoku erfolgte im Mai 1979 in der Zeitschrift 'Dell Pencil Puzzles \& Word Games', allerdings unter dem Namen 'Number Place', wie man \cite{Wolf2014} entnehmen kann.\\
Die Regeln sind sehr simpel, denn es gibt nur eine. Sudoku ist daher sehr einfach zu lernen. Dennoch kann ein Spieler nicht jedes Sudoku auf Anhieb lösen, da für manche Sudokus sehr komplexe Techniken nötig sind. Das Erlernen der Techniken und das Finden der benötigten Muster in den Zahlen stellt beim Meistern des Spiels die größte Herausforderung dar. Manche Spieler versuchen auch, Sudokus möglichst schnell zu lösen.\\
Verschiedene Sudokus werden von Spielern als unterschiedlich schwer empfunden, daher ist annähernd jedes veröffentliche Sudoku mit einem Schwierigkeitsgrad versehen. Je nachdem, wo Sudokus auftauchen, unterscheiden sich auch die Schwierigkeitsgrade, obwohl sie manchmal sogar den selben Namen haben. Das liegt daran, dass es kein vereinheitlichstes Vorgehen bei der Bewertung des Schwierigkeitsgrades gibt. Daher haben sich eine Menge unterschiedliche Vorgehensweisen und Bewertungsskalen entwickelt. Manche Skalen haben nur die Einträge 'Leicht', 'Mittel' und 'Schwer', andere unterteilen Sudokus in sieben verschiedene Schwierigkeitsstufen.\\
Der Spielspaß ist aber stark davon abhängig, dass ein Spieler die zu ihm passende Schwierigkeitsstufe findet. Wenn ein Spieler ein zu leichtes Sudoku löst, dann stellt es keine Herausforderung dar und er hat kein Erfolgserlebniss, wenn er fertig ist. Ist das Sudoku allerdings zu schwer, dann kommt schnell ein Gefühl der Frustration auf, da keine Zahlen gefunden werden, die in das Spielfeld eingetragen werden können.\\
Die verschiedenen Skalen der unterschiedlichen Veröffentlichungsorte machen einen Wechsel zum Beispiel von der Tageszeitung zu einem Buch unnötig kompliziert. Spieler müssen wieder den zu ihnen passenden Schwierigkeitsgrad finden. Daher befasst sich diese Arbeit unter anderem damit, wie zwei Skalen miteinander verglichen werden können.\\
Sudokus sind für die Bearbeitung mit Computern sehr gut geeignet. Da das Spielfeld aus 81 Felder besteht, die jeweils 10 verschiedene Einträge haben können\footnote{leer oder eine der Ziffern 1 bis 9}, ist die interne Repräsentation in einem Computerprogramm sehr leicht. Weiterhin gibt es keine anderen Spieler, keinen Zufall und die Informationen über das Spiel sind immer vollständig.\\
Der neue Ansatz dieser Arbeit ist das Einteilen von Sudokus in Schwierigkeitsgrade mit Methoden des maschinellen Lernens. Dabei soll eine Methode entwickelt werden, die aus einem Sudoku Informationen gewinnt und mit Hilfe dieser Informationen eine Aussage über den Schwierigkeitsgrad treffen kann. Ausserdem soll festgestellt werden, welche Informationen zur Einteilung relevanter sind als andere.