\section{Struktur}
Nach der Einleitung in Kapitel \ref{Einleitung} werde ich einen Überblick über die Grundlagen geben, auf denen diese Arbeit aufbaut. Zuerst werde ich in Kapitel \ref{Regeln} auf die Regeln für Sudokus eingehen. Das beinhaltet die Regeln, die der Spieler beachten muss und auch die Regeln, die ein Sduoku erfüllen muss, um als Sudoku zu gelten. \\
Anschließend werden in Kapitel \ref{Begriffe} einige Begriffe definiert, die im weiteren Verlauf der Arbeit auftauchen werden.\\
Danach werde ich in Kapitel \ref{Einteilungen} beschreiben, welche Ansätze es zur Ermittlung des Schwierigkeitsgrades bereits gibt und auf deren Vor- und Nachteile eingehen.\\
Da, für die in dieser Arbeit vorgestellte Methode, Lösungstechniken für Sudokus essentiell sind, werde ich in Kapitel \ref{Methoden} die verwendeten Lösungsmethoden erklären. Sie sind innerhalb des Kapitels nach ihrer Schwierigkeit geordnet. Das sind nicht alle existierenden Lösungsmethoden, aber Weitere wurden im Rahmen dieser Arbeit nicht benötigt.\\
In Kapitel \ref{Methode} möchte ich die von mir entwickelte Methode vorstellen. Dazu werde ich in Kapitel \ref{Klassifikation} darauf eingehen, was Klassifikation im Kontext von maschinellem Lernen bedeutet und wozu Klassifikation in der Methode benötigt wird.\\
Anschließend wird in Kapitel \ref{Merkmalsextrahierung} erklärt, was ein Featurevector ist und wozu er in dieser Arbeit verwendet wird.\\
In Kapitel \ref{Aufbau} werde ich darauf eingehen, wie genau der Featurevector aus dem Sudoku gewonnen wird und welche Eintrage darin enthalten sind.\\
Das Kapitel \ref{Entkopplung} beschäftigt sich mit einer sehr nützlichen Optimierung des Featurevectors, die speziell Sudokus betrifft und die im Rahmen dieser Arbeit entwickelt wurde.\\
Um die entwickelte Methode zu testen, wurde Software entwickelt, die in Kapitel \ref{Eigene_Software} beschrieben wird. Da einige Funktionalität bereits in open source Projekten implementiert wurde, wurden diese Projekte bei der Entwicklung der Software verwendet. Welchen Zweck diese Projekte innerhalb des selbst entwickelten Programms erfüllt, wird in Kapitel \ref{Fremdsoftware} erklärt.\\
Um die Methode zu testen, werden neben einer Implementierung auch Testdaten benötigt. Welche Testdaten verwendet wurden und woher diese stammen, wird in Kapitel \ref{Trainingsdaten} erklärt.\\
Die Ergebnisse der Tests, werde ich in Kapitel \ref{Ergebnisse} vorstellen. Dazu zählen die Güte der Klassifikation, der Einfluss der Parameter der Klassifiikationsalgorithmus und der Einfluss der verwendeten Lösungsmethoden. Weiterhin werde ich in diesem Kapitel auf die Ergebnisse der Methode zum Vergleich verschiedener Bewertungsskalen für den Schwierigkeitsgrad von Sudokus eingehen.\\
Das letzte Kapitel (\ref{Zusammenfassung}) fasst die Arbeit zusammen und gibt einen Ausblick auf weitere, mögliche Forschung.