\newpage
\section{Struktur}
Nach der Einleitung in Kapitel \ref{Einleitung} wird ein Überblick über die Grundlagen gegeben, auf denen diese Arbeit aufbaut. Zuerst wird in Kapitel \ref{Regeln} auf die Regeln für Sudokus eingegangen. Das beinhaltet die Regeln, die der Spieler beachten muss und auch die Regeln, die ein Sudoku erfüllen muss, um als Sudoku zu gelten. \\
Anschließend werden in Kapitel \ref{Begriffe} einige Begriffe definiert, die im weiteren Verlauf der Arbeit auftauchen werden.\\
Danach wird in Kapitel \ref{Einteilungen} beschrieben, welche Ansätze es zur Ermittlung des Schwierigkeitsgrades bereits gibt und auf deren Vor- und Nachteile eingehen.\\
Da, für die in dieser Arbeit vorgestellte Methode, Lösungstechniken für Sudokus essentiell sind, werden in Kapitel \ref{Methoden} die verwendeten Lösungsmethoden erklärt. Sie sind innerhalb des Kapitels nach ihrer Schwierigkeit geordnet. Das sind nicht alle existierenden Lösungsmethoden, aber Weitere wurden im Rahmen dieser Arbeit nicht benötigt.\\
In Kapitel \ref{Methode} wird die entwickelte Methode zur Ermittlung des Schwierigkeitsgrades vorgestellt. Dazu wird in Kapitel \ref{Klassifikation} darauf eingegangen, was Klassifikation im Kontext von maschinellem Lernen bedeutet und wozu Klassifikation in der vorgestellten Methode benötigt wird.\\
Anschließend wird in Kapitel \ref{Aufbau} erklärt, was ein Featurevektor ist, wie er ermittelt wird und wie diese Arbeit Featurevektoren verwendet.\\
Das Kapitel \ref{Entkopplung} beschäftigt sich mit einer sehr nützlichen Optimierung des Featurevektors, die speziell Sudokus betrifft und die im Rahmen dieser Arbeit entwickelt wurde.\\
Um die entwickelte Methode zu testen, wurde Software entwickelt, die in Kapitel \ref{Eigene_Software} beschrieben wird. Da einige Funktionalität bereits in open source Projekten implementiert wurde, wurden diese Projekte bei der Entwicklung der Software verwendet. Welchen Zweck diese Projekte innerhalb des selbst entwickelten Programms erfüllen, wird in Kapitel \ref{Fremdsoftware} beschrieben.\\
Um die Methode zu testen, werden neben einer Implementierung auch Testdaten benötigt. Welche Testdaten verwendet wurden und woher diese stammen, wird in Kapitel \ref{Trainingsdaten} erklärt.\\
Die Ergebnisse der Tests werden in Kapitel \ref{Ergebnisse} vorgestellt. Dazu zählen die Güte der Klassifikation, der Einfluss der Parameter des Klassifikationsalgorithmus und der Einfluss der verwendeten Lösungsmethoden. Weiterhin wird in diesem Kapitel auf die Ergebnisse der Methode zum Vergleich verschiedener Bewertungsskalen für den Schwierigkeitsgrad von Sudokus eingegangen.\\
Das letzte Kapitel (\ref{Zusammenfassung}) fasst die Arbeit zusammen und gibt einen Ausblick auf weiterführende Forschungsansätze.