\newpage
\section{Ergebnisse des Mappings}
Wie bereits in Kapitel \ref{Mapping} beschrieben, wurde ein Mapping zwischen zwei Skalen durchgeführt. Hierzu werden Sudokus aus zwei Quellen benötigt. Als erstes Set werden wieder die 5000 Sudokus aus fünf Schwierigkeitsgraden verwendet. Das zweite Set besteht aus Ingesammt 224 Sudokus, jeweils 32 Sudokus in sieben Schwierigkeitsstufen. Zunächst wird der Klassifikator mit dem ersten Set trainiert. Er kennt daher die fünf Klassen \textbf{Easy}, \textbf{Middle}, \textbf{Hard}, \textbf{Unfair} und \textbf{Extreme}. In diese Klassen werden nun die Sudokus des zweiten Sets durch den Klassifikator eingeteilt. Die Ergebnisse lassen sich sehr gut als Matrix darstellen.\\
\begin{figure}[H]
\centering
\begin{tabular}{ l | l |  c  c  c  c  c |}
\multicolumn{7}{c}{\textbf{Zugeordnete Klasse}}\\
\cline{2-7}
\multirow{6}{*}{\begin{turn}{90}\textbf{Ursprüngliche Klasse}\end{turn}}
&  & Easy & Middle & Hard & Unfair & Extreme\\
\cline{2-7}
& Sehr Einfach & 32 & 0 & 0 & 0 & 0\\
& Einfach & 32 & 0 & 0 & 0 & 0\\
& Standard & 32 & 0 & 0 & 0 & 0\\
& Moderat & 0 & 32 & 0 & 0 & 0\\
& Anspruchsvoll & 0 & 1 & 30 & 1 & 0\\
& Sehr Anspruchsvoll & 0 & 1 & 28 & 2 & 1\\
& Teuflisch & 0 & 0 & 7 & 8 & 17\\
\cline{2-7}
\end{tabular}
\caption{Mapping von zwei Bewertungsskalen}
\end{figure}
In \textbf{Abbildung 5.8} sind die Ergebnisse des Mappings als Matrix dargestellt. Man sieht, dass die ersten drei Klassen \textbf{Sehr Einfach}, \textbf{Einfach} und \textbf{Standard} auf der zweiten Skala alle der Klasse \textbf{Easy} auf der ersten Skala zugeordnet werden. Man kann der Matrix weiter entnehmen, dass die Schwierigkeit der Klasse \textbf{Moderat} genau der Schwierigkeit der Klasse \textbf{Middle} entspricht. Das selbe gilt auch für die Klassen \textbf{Anspruchsvoll} und \textbf{Hard}.
Die Klasse \textbf{Sehr Anspruchsvoll} ist etwas schwerer als die Klasse \textbf{Hard}, diesen Schluss legen die drei Sudokus der Matrix nahe, die schwerer als \textbf{Hard} klassifiziert wurden.\\
Die meißten Sudokus der Klasse \textbf{Teuflisch} wurden als \textbf{Extreme} klassifiziert. Allerdings sind auch fast 50\% der Sudokus leichter klassifiziert worden, daher ist die Klasse \textbf{Teuflisch} im Hinblick auf den Schwierigkeitsgrad zwischen \textbf{Unfair} und \textbf{Extreme} einzuordnen. Generell kann man sagen, dass die Skala 2 eine genauere Unterscheidung zwischen den leichteren Klassen trifft, da die drei einfachsten Klassen von Skala 2 alle der einfachsten Klasse von Skala 1 zugeordnet wurden.