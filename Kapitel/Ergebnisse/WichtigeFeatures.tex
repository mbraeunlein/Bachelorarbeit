\section{Relative Güte der Features}
Der Featurevector besteht in seiner ungekürzten Form aus 261 Features. Im J48 Klassifizierer wird beim Aufbau des Entscheidungsbaumes jedes Feature als mögliche Alternative für jeden Knoten im Baum in Betracht gezogen. Hierdurch entsteht eine hohe Laufzeit. Es ist möglich, die Auftrittshäufigkeiten von Featurs im Entscheidungsbaum zu ermitteln. Dazu werden zufällige Untermengen der Trainingsmenge erzeugt, in denen die Klassenverteilung erhalten bleibt. Mit jeder Untermenge wird ein Klassifizierer trainiert. Die entstandenen Entscheidungsbäume der Klassifizierer werden dann durchlaufen und für jedes Feature wird die Anzahl der Vorkommen im Entscheidungsbaum gespeichert. Features, die viele Informationen für den Klassifizierer liefern, kommen häufiger im Baum vor als andere.