\newpage
\section{Existierende Einteilungsschemata}
\label{Einteilungen}
Für die Zuordnung eines Schwierigkeitsgrades zu einem Sudoku existieren bereits verschiedene Methoden. Die einfachsten orrientieren sich nur an Merkmalen, die auf den ersten Blick sichtbar sind, wie etwa die Anzahl der vorgegebenen Ziffern. Diese Methoden haben den Vorteil, dass das Sudoku bewertet werden kann, ohne es zu lösen. Ausserdem sind sie sehr einfach zu implementieren und auch sehr schnell. Die Genauigkeit solcher Methoden ist allerdings nicht besonders hoch, da die offensichtlichen Merkmale nicht mit einbeziehen, wie schwer das Lösen eines Sudokus ist, da es ja während der Bewertung des Sudokus nicht gelöst wird und somit dem Lösungsweg kein Schwierigkeitsgrad zugeordnet werden kann.\\
Daher ist es sinnvoll, das Sudoku bei der Bewertung zu lösen und die Schwierigkeit des Lösungsweges zu schätzen. Ein Lösungsweg besteht aus einer Reihe von Lösungsschritten. Mit jedem Schritt erfährt der Spieler mehr über das Sudoku, das bedeutet, dass er eine Ziffer einsetzen kann oder dass er ausschließt, dass eine Ziffer in einem bestimmten Feld stehen kann. Gegebenenfalls kann er mit einem Lösungsschritt sogar mehrere Ziffern in einem Feld, oder eine Ziffer in mehreren Felder ausschließen. Sehr komplexe Methoden erlauben auch den gleichzeitigen Ausschluss von mehreren Ziffern in mehreren Feldern.\\
Es gibt also verschiedene Arten von Lösungsschritten, diese haben auch eine verschiedene Schwierigkeit. Dabei entscheidend ist für diese Arbeit die vom menschlichen Spieler empfundene Schwierigkeit und nicht etwa die Komplexität der Implementierung oder die Laufzeit der Lösungsmethode auf einem Rechner, da die Bewertung später als Orrientierung für Menschen dienen soll.\\
Die Website \url{http://sudoku.soeinding.de/strategie/strategie03.php} ist ein Beispiel für ein solches Vorgehen.\\
Eine weitere Methode ist die Messung der Zeit, die menschliche Spieler zum Lösen von Sudokus brauchen. Hierbei ist das Problem, dass die Klassifikation eines Sudokus sehr lange dauert, da es erst von mehreren Spielern gelöst werden muss. Ausserdem ist die Erfahrung und damit die Spielstärke von Sudoku Spielern sher unterschiedlich und sollte in die Bewertung mit einfließen.\\
Ein Beispiel für diesen Ansatz der Klassifikation ist diese Website \url{http://sudokugarden.de/de/online/dstats}.\\
In dieser Arbeit wird versucht, ein Einteilungsschma zu finden, dass auf der Schwierigkeit des Lösungsweges beruht, aber auch offensichtliche Merkmale einbezieht.