\newpage
\section{Backtracking}
\label{Backtracking}
\textit{Backtracking} arbeitet nach dem \textit{trial and error} Prinzip. Es wird eine zufällige Zahl aus der Kandidatenliste eines Feldes in das Feld eingesetzt. Danach werden die Kandidatenlisten wieder aktualisiert und Backtracking beginnt von vorne. Wenn bemerkt wird, dass durch das Einsetzen einer Zahl eine Situation entsteht, die die Sudoku Regel verletzt, dann wird der letzte Schritt zurück genommen. \textit{Backtracking} ist also ein rekursiver Algorithmus und führt eine Tiefensuche über den Lösungsraum des Sudokus durch, der als Baum dargestellt werden kann. Sobald ein Ast des Baumes komplett durchsucht wurde, ohne eine Lösung zu finden, dann wird im Baum so lange wieder nach oben gegangen, bis eine andere Abzweigung verfügbar ist. Wenn eine Lösung gefunden wurde, dann terminiert der Algorithmus. Die Terminierung ist also immer gegeben, da jedes Sudoku eine eindeutige Lösung hat, wie in den \textit{Regeln} \ref{Regeln} definiert.\\
Backtracking unterscheidet sich von den anderen Lösungstechniken dadurch, dass es das Sudoku vollständig löst, unabhängig davon, wie viele Zahlen dafür eingetragen werden müssen.