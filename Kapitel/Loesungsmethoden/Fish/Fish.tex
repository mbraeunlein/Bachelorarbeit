\newpage
\section{Fish}
Die \textit{Fish} Methoden sind ein Sammelbegriff für eine ganze Grupppe von Methoden, die alle nach dem gleichen Prinzip arbeiten. Wie echte Fische hat dieses Prinzip eine sehr große Anzahl Unterarten hervorgebracht. Kleine Fische wie zum Beispiel X-Wing sind von geübten Sudoku Spielern noch zu finden, wenn die Fische allerdings größer werden, dann sind sie nur noch mit sehr hohem Aufwand manuell zu finden und daher eher zur Verarbeitung mit dem Computer gedacht. \\
Auf einer Internetseite, die sich unter anderem mit den Lösungsmethoden für Sudokus befasst, findet sich die folgende Erklärung zur Funktionsweise von Fischen.

\begin{quote}[...] Man suche eine bestimmte Anzahl von Häusern, die sich nicht überschneiden. Diese Häuser werden als Base-Sets (Basismengen) bezeichnet (Set wird hier synonym für Haus verwendet), die in diesen Häusern enthaltenen Kandidaten sind die Basiskandidaten. Nicht überschneiden bedeutet hier, dass kein Basiskandidat in mehr als einem Haus enthalten sein darf, die Häuser selbst dürfen sich schon überlappen. Nun suche man eine gleiche Anzahl an sich nicht überschneidenden Häusern, die alle Basiskandidaten abdecken (engl.: cover). Diese neuen Häuser sind die Cover-Sets, sie enthalten die Coverkandidaten. Wenn eine solche Kombination existiert, können alle Coverkandidaten gelöscht werden, die nicht gleichzeitig Basiskandidaten sind.
\footnote{Quelle: \url{http://hodoku.sourceforge.net/de/tech_fishg.php}}
\footnote{Häuser stehen hier für Figuren}
\end{quote}