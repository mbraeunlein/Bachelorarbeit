\chapter{Merkmalsextrahierung}
Um Sudokus mit Hilfe von Weka nach ihrem Schwierigkeitsgrad zu klassifizieren, ist es nötig, Featurevektoren aus den Sudokus zu extrahieren. Ein Featurevektor repräsentiert ein Sudoku, ein Eintrag des Featurevektors steht für eine Eigenschaft des Sudokus. Einen Eintrag des Featurevektors nennt man ein Feature.\\
Also ist die Frage: Welche Features hat ein Sudoku? Genauer gesagt werden Features gesucht, aus denen man Rückschlüsse auf den Schwierigkeitsgrad eines Sudokus ziehen kann.\\
Man kann ein solches Feature bereits finden, ohne überhaupt einen einzigen Lösungsschritt durchgeführt zu haben. Es handelt sich um die Anzahl der vorgegebenen Ziffern. Je mehr Ziffern vorgegeben sind, desto weniger muss der Spieler selbst finden und umso einfacher sollte das Sudoku für ihn werden. Aus der Anzahl der vorgegebenen Ziffern lassen sich aber noch mehr Informationen gewinnen. Ist von einer bestimmten Ziffer zu Anfang keine Position bekannt, dann wird das Sudoku als schwerer empfunden. Sind dagegen von jeder Ziffer annähernd gleich viele Positionen vorgegeben, dann wird das Sudoku für den Spieler meißt einfacher. Daher werden zu jeder Ziffer die Anzahl der vorkommenden Positionen im Featurevektor gespeichert.\\
In Kaptiel \ref{Kandidatenlisten} wurde auf Kandidatenlisten eingegangen. Erstellt man nun zu Spielbeginn eine Kandidatenliste für jedes Feld, so kann 