\section{Klassifikation}
\label{Klassifikation}
Unter einer Klassifikation versteht man in der Informatik das Einteilen von Objekten in vorher festgelegte Klassen. Diese Einteilung wird von einem Algorithmus durchgeführt, der anhand von festgelegten Merkmalen jedem Objekt eine Klasse zuordnet. Einen solchen Algorithmus nennt man Klassifikator. Um die Qualität eines Klassifikators zu analysieren gibt es verschiedene Metriken.
\begin{itemize}
\item Accuracy - Die Anzahl der richtig zugeordneten Klassen
\item Recall - Der Anteil der positiven Beispiele, die auch positiv klassifiziert wurden
\item Precision - Der Anteil der positiv klassifizierten Beispiele, die auch positiv sind
\end{itemize}
Ein Klassifikator benötigt vor der Phase der Klassifikation aber zunächst einmal eine Trainigsphase, in der er anhand von Beispielen lernt. Mit dem erlenten Wissen wird anschließend die Einteilung in die Klassen vorgenommen.\\
Es gibt viele verschiedene Ansätze für Klassifikatoren, von denen die wichtigsten in einem open source Framework implementiert sind. Dieses Framework heisst Weka\footnote{\url{http://www.cs.waikato.ac.nz/ml/weka/index.html}} und wurde im praktischen Teil dieser Bachelorthesis verwendet.\\
Weka arbeitet unter anderem mit dem .arff\footnote{\url{http://weka.wikispaces.com/ARFF}} Format. In einer .arff Datei befindet sich neben den Metadaten hauptsächlich eine Sammlung von Featurevektoren. Jeder Featurevektor beschreibt ein zu klassifizierendes Objekt. Ein Eintrag in einem Featurevektor beschreibt eine Eigenschaft des beschriebenen Objekts. Das könnte bei Fahrzeugen zum Beispiel die Anzahl der Reifen sein. Bezogen auf Sudokus bedeutet das, dass jedes Sudoku durch einen Featurevektor beschrieben wird. Jede zur Klassifikation verwendete Eigenschaft eines Sudokus ist dann ein Wert im entsprechenden Featurevektor. Die Schwierigkeitsgrade der Sudokus sind die vorgegebenen Klassen.\\
Jeder Klassifikator in Weka hat als Eingabe eine Liste von Featurevektoren. Möchte man also das Zuordnen von Sudokus zu Schwierigkeitsgraden mit Weka realisieren, dann muss eine Methode entwickelt werden, die aus einem gegebenen Sudoku einen Featurevektor extrahiert.\\
Bei der Klassifikation wird ein Verfahren angewendet, das als \textit{cross validation} bekannt ist und auch von Weka zur Verfügung gestellt wird. Dabei werden die Daten in eine vorgegebene Anzahl gleich großer Folds eingeteilt. Ein Fold ist eine Sammlung von Featurevektoren. Wenn die Einteilung in k Folds erfolgt ist, dann wird der Klassifikator k mal ausgewertet, einmal mit jedem Fold.\\
Die Qualität der resultierenden Klassifikation ist sehr stark von der Wahl der Einträge des Featurevektors abhängig. Daher lag das Hauptaugenmerk dieser Bachelorarbeit auf dem Herausarbeiten der passenden Einträge für den Featurevektor, der das Sudoku beschreibt. Später wird aber auch auf die Wahl des Klassifikators und die Optimierung der Parameter eingegangen.\\