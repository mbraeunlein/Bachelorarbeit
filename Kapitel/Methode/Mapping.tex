\newpage
\section{Mapping verschiedener Skalen}
\label{Mapping}
Die Schwierigkeitsstufen von Sudokus aus verschiedenen Quellen sind, nach dem Empfinden von menschlichen Spielern, nicht immer identisch. Verschiedene Quellen haben verschiedene Skalen zur Bewertung des Schwierigkeitsgrades, deren Einteilungen die Sudokus nach unterschiedlichen Maßen zugeordnet werden. Daher ist es nützlich, verschiedene Skalen miteinander vergleichen zu können und ähnlich schwere Kategorien einander zuzuordnen. Im Folgenden wird eine Methode vorgestellt, die auf der Klassifikation des Schwierigkeitsgrades der Sudokus mit Hilfe von maschinellem Lernen aufbaut und es ermöglicht, einen Vergleich zwischen zwei unterschiedlichen Skalen zu erstellen.\\
Angenommen es gibt zwei Sets von Sudokus, $\mathbf{Set 1}$ und $\mathbf{Set 2}$. Diese haben zwei unterschiedliche Skalen, $\mathbf{Skala 1}$ und $\mathbf{Skala 2}$, für die Schwierigkeitsgrade. Nun wird ein Klassifizierer mit $\mathbf{Set 1}$ trainiert. Der Klassifizierer kann nun beliebige Sudokus den Klassen aus $\mathbf{Skala 1}$ zuordnen. Das wird benutzt, um die Sudokus aus $\mathbf{Set 2}$ zu klassifizieren. Zu jedem Sudoku aus $\mathbf{Set 2}$ ist nun die ursprüngliche Klasse aus $\mathbf{Skala 2}$ bekannt und zusätzlich auch die vom Klassifizierer zugeordnete Klasse auf $\mathbf{Skala 1}$. Nun wird für jede Klasse in $\mathbf{Skala 2}$ ausgewertet, welcher Klasse aus $\mathbf{Skala 1}$ die meißten Sudokus zugeordnet werden. Daraus lässt sich schließen, welchem Schwierigkeitsgrad auf $\mathbf{Skala 1}$ die jeweilige Klasse am ehesten entspricht. Wenn das Ergebniss sehr uneindeutig ist, zum Beispiel werden 40\% $\mathbf{Klasse 1}$ zugeordnet und 60\% werden $\mathbf{Klasse 2}$ zugeordnet, dann ist der Schluss, dass die Schwierigkeit der entsprechenden Klasse zwischen $\mathbf{Klasse 1}$ und $\mathbf{Klasse 2}$ liegt.\\
Um diese Vorgehensweise zu verdeutlichen, wird das folgende Beispiel betrachtet. Es gibt Sudokus mit bekanntem Schwierigkeitsgrad aus zwei Quellen. Diese Quellen verwenden zur Bewertung des Schwierigkeitsgrades zwei unterschiedliche Skalen. Die erste Quelle teilt Sudokus in die Schwierigkeitsgrade \textbf{Leicht}, \textbf{Mittel} und \textbf{Schwer} ein, die zweite Quelle in \textbf{Easy}, \textbf{Medium}, \textbf{Hard} und \textbf{Very Hard}. Der Klassifizierer wird mit den Sudokus aus der ersten Quelle trainiert und kennt daher nur die Bewertungen für die Schwierigkeitsgrade dieser Quelle. Die Einteilung von Sudokus in die Klassen der zweiten Quelle wurde nicht trainiert. Nun wird der Klassifizierer verwendet, um Sudokus aus der zweiten Quelle zu klassifizieren. Als Ergebniss der Klassifikation erhält man für jedes Sudoku der zweiten Quelle einen Schwierigkeitsgrad aus der Skala der ersten Quelle. Angenommen die Klassifikation mit jeweils 100 Sudokus aus jeder Klasse von Quelle 2 liefert das, in \textbf{Tabelle 3.1} dargestellte, Ergebniss.\\
\begin{table}[h]
\centering
\begin{tabular}{ l | l | c c c |}
\multicolumn{5}{r}{\textbf{Zugeordnete Klasse}}\\
\cline{2-5}
\multirow{5}{*}{\begin{turn}{90}\textbf{\textbf{\begin{tabular}{@{}c@{}}Ursprüngliche\\Klasse\end{tabular}}}\end{turn}}
& & Leicht & Mittel & Schwer\\
\cline{2-5}
& Easy & 90 & 8 & 2 \\
& Medium & 50 & 40 & 10 \\
& Hard & 2 & 78 & 20\\
& Very Hard & 0 & 10 & 90\\
\cline{2-5}
\end{tabular}
\caption{Beispiel Resultat}
\end{table}
\newline
Da 90\% aller Sudokus aus der Klasse \textbf{Easy} der Klasse \textbf{Leicht} zugeordnet wurden, ist es eindeutig, dass diese beiden Klassen annähernd gleich schwer sind. Es gibt hier keine andere Klasse, die der Klasse \textbf{Easy} mehr entspricht. Bei der Klasse \textbf{Medium} lässt sich keine klare Aussage treffen, da die Ergebnisse mit 50\% für \textbf{Leicht} und 40\% für \textbf{Mittel} sehr nahe aneinander liegen. Die Schwierigkeit der Klasse \textbf{Medium} liegt zwischen den Schwierigkeiten dieser beiden Klassen. Die Klasse \textbf{Hard} entspricht am ehesten der Klasse \textbf{Mittel}, allerdings ist eine leichte Tendenz zur Klasse \textbf{Schwer} zu erkennen. Daher ist \textbf{Hard} etwas schwerer als \textbf{Mittel}. Die Klasse \textbf{Very Hard} ist eindeutig der Klasse \textbf{Schwer} zuzuordnen, da 90\% aller Sudokus dementsprechend klassifiziert wurden.