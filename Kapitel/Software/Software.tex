\chapter{Software}
Um die Ergebnisse der Klassifikation ermitteln zu können, wurde im Rahmen dieser Bachelorarbeit eine Software entwickelt, die es erlaubt, Sudokus aus Dateien einzulesen und den Featurevector zu extrahieren. Mit den Featurevectoren wird dann einen Klassifizierer trainiert und evaluiert. Ausserdem erlaubt es die Software, dass ein Mapping zwischen zwei verschiedenen Bewertungsskalen vorgenommen wird. Darauf werde ich im Kapitel \textit{Eigene Software} \ref{Eigene_Software} eingehen.\\
Die ganze Software ist in Java geschrieben, der Quellcode ist bei der Abgabe beigelegt. Ausserdem ist die Software momentan verfügbar auf \url{https://github.com/mbraeunlein/ExtendedHodoku}. Die Software lässt sich einfach über ein Terminal starten\\
\textit{java -jar ExtendedHodoku cross trainSudokus.txt}\\
Der erste Parameter der Anwendung ist entweder \textit{cross}, um corss validation auszuführen, \textit{test}, um den Klassifikator mit einem Trainingsset und einem Testset zu evaluieren, oder \textit{map}, um ein Mapping zwischen Sets mit unterschiedlichen Skalen zu erstellen.