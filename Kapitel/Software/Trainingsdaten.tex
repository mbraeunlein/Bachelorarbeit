\section{Trainingsdaten}
\label{Trainingsdaten}
Um fundierte Aussagen über die Qualität der Klassifikation treffen zu können, wird eine große Menge an Trainingsdaten benötigt. Diese müssen bereits vollständig in Schwierigkeitsstufen eingeteilt worden sein. Das ist nötig, da der Klassifikator eine Schwierigkeitsstufe zuordnet und die Qualität der Zuordnung evaluiert werden soll. Um also festzustellen, ob der Klassifikator die richtige Klasse zugeordnet hat, muss diese bekannt sein.\\
Kostenlose und frei verfügbare Sudokus in digitaler Form mit definiertem Schwierigkeitsgrad lassen sich nicht leicht finden. Daher habe ich bei einigen großen Zeitungen, aufs deren Websites Sudokus zu finden waren, nachgefragt, ob es möglich ist, ihre Sudokusammlungen zur Verfügung zu stellen. Die Anfragen wurden aber leider abgelehnt. Auf eine Anfrage an die Website \url{http://sudoku.soeinding.de/} wurden von sieben Schwierigkeitsgraden jeweils 32 Sudokus bereitgestellt. Da diese Trainigsdaten nicht ausreichten wurden mit dem open source Programm Hodoku\footnote{\url{http://hodoku.sourceforge.net/de/index.php}} jeweils 1000 Sudokus von fünf unterschiedlichen Schwierigkeitsgraden generiert.\\
Wie schon an der Anzahl der Schwierigkeitsstufen zu erkennen ist, unterscheiden sich die Skalen der beiden Quellen. Daher konnten die Sudokus nicht gemeinsam klassifiziert werden. Auch war es nicht möglich, eine Quelle als Trainingsdaten für den Klassifikator zu verwenden um ihn anschließend mit der anderen Menge auszuwerten. Allerdings kann man eine Verbindung zwischen den Skalen suchen, zum Beispiel Klassen mit gleich schweren Sudokus.