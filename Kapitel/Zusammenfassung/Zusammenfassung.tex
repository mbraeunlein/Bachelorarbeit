\chapter{Zusammenfassung und Ausblick}
\label{Zusammenfassung}
Im Rahmen dieser Arbeit wurde die Klassifikation der Schwierigkeitsgrade von Sudokus mit Methoden des maschinellen Lernens erarbeitet, implementiert und anschließend ausgewertet. Nach einem kurzen Überblick über die Regeln und eine Einführung in die Begriffe, wurden ausführlich die verwendeten Lösungsmethoden beschrieben und an Beispielen verdeutlicht. Im Anschluss wurde die entwickelte Methode vorgestellt. Das beinhaltet eine kurze Einführing in Klassifikationsverfahren, das Extrahieren der Featurevektoren aus Sudokus und die Beschreibung der zum Mapping angewandten Vorgehensweise. Danach wurde beschrieben, wie die Methode in Software umgesetzt wurde und welche Trainigsdaten zur Evaluation benutzt wurden. Zum Schluss wurde ausführlich diskutiert, welche Ergebnisse mit der vorgestellten Methode erzielt wurden und wie diese zu interpretieren sind. Ausserdem wurde der Informationsgewinn der Features für den Klassifizierer ermittelt und eine Bewertung der Qualität der Features vorgenommen.\\
Das wichtigste Ergebniss der Arbeit ist, dass der meißte Informationsgehalt über den Schwierigkeitsgrad der Sudokus in den verwendeten Lösungsmethoden zu finden ist. Es ist also nicht sinnvoll möglich, den Schwierigkeitsgrad eines Sudokus zu ermitteln, ohne es zu lösen. Bedingt dadurch, dass die Daten in den Testsets ebenfalls auf Basis der Lösungsmethoden eingeteilt wurden, ist dieses Ergebniss differenziert zu sehen. Um es zu untermauern, wird ein Testset benötigt, dass von menschlischen Spielern bewertet wurde. Alle Sudokus in den gefundenen Quellen wurde allerdings maschinell in die entsprechenden Schwierigkeitsgrade eingeteilt, daher ist das Ergebniss für den empfundenen Schwierigkeitsgrad bei menschlichen Spielern nicht sehr aussagekräftig. Gerade die Tatsache, dass kaum Sudokus von Hand eingeteilt werden, macht das Ergebniss aber wieder wertvoll, denn es wurde für Sudokus ermittelt, die von Computern eingeteilt wurden und trifft damit auf die meißten Quellen von Sudokus zu.\\
Zusätzlich war es möglich, eine Verbindung zwischen verschiedenen Bewertungsskalen herzustellen. Wenn man über zwei Sets von Sudokus verfügt, die aus unterschiedlichen Quellen stammen und auch nur mit dem Schwierigkeitsgrad der Ursprungsquelle versehen sind, dann ist es möglich, den jeweils anderen Schwierigkeitsgrad zu ermitteln. Das ist von Nutzen, um verschiedene Sudokuskalen zu vereinheitlichen.\\
Weitere Forschungsarbeit ist nötig, um die Qualität des Mappingverfahrens sicher zu stellen. Um eine Aussage über dessen Qualität treffen zu können, ist es nötig, ein Testset zu haben, bei dem beide Schwierigkeitsgrade bereits zugeordnet sind. Dann kann überprüft werden, wie viele Sudokus vom Mappingverfahren korrekt zugeordnet werden. Um die Ungenauigkeit der Klassifikation zwischen schwierigeren Klassen zu beheben, können schwerere Lösungsmethoden ausprobiert werden. Ausserdem kann getestet werden, wie sich die Klassifikationsgenauigkeit bei der Verwendung verschiedener Klassifikationsmethoden verhält. Es wäre auch denkbar, ganz neue Arten von Features zu entwickeln, wie etwa die Zeit, die ein menschlicher zum Lösen des Sudokus benötigt.